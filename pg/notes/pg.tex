\input{/Users/daniel/github/config/preamble-por.sty}%available at github.com/danimalabares/config
%\input{/Users/daniel/github/config/thms-por.sty}%available at github.com/danimalabares/config

\newcommand{\rightlooparrow}{\mathbin{
    \vbox{\openup-10.25pt\halign{\hss$##$\hss\cr\circ\cr\longrightarrow\cr}}
}}

\begin{document}
\bibliographystyle{alpha}

\begin{minipage}{\textwidth}
	\begin{minipage}{1\textwidth}
		\hfill Daniel González Casanova Azuela
		
		{\small Prof. Henrique Bursztyn\hfill\href{https://github.com/danimalabares/sg}{github.com/danimalabares/sg}}
	\end{minipage}
\end{minipage}\vspace{.2cm}\hrule

\vspace{10pt}
{\huge Geometria de Poisson}
\tableofcontents
\section{Aula 1}

Livros:
\begin{itemize}
\item Lectures em PG, Carnic-Gernandes-Marcit
\item A brief introduction to PG, HB
\end{itemize}

\subsection{História}

Em 1809, Poisson buscava uma formulação geométrica da mecânica clássica (celeste). Espaço fase, função Hamiltoniana, campo Hamiltoniano, equações de Hamilton. O primeiro colchete de Poisson é
\begin{align*}
	\{\cdot ,\cdot \}: C^\infty(\mathbb{R}^{2n})\times C^\infty(\mathbb{R}^{2n}) &\longrightarrow C^\infty(\mathbb{R}^{2n}) \\
	\{f,g\} &=\sum_{i=1}^n \frac{\partial f}{\partial p_i}\frac{\partial g}{\partial q_i}-\frac{\partial f}{\partial q_i}\frac{\partial g}{\partial p_i}
\end{align*}

Interpretação dinâmica: \(\{H,f\}=\mathcal{L}_{X_H}f\), que como temos antisimetria, é o mesmo que \(-\mathcal{L}_{X_f}H\).

Conhecer a dinâmica do sistema é resolver às equações de Hamilton, EDOs. Integrais primeiras. Aí nasce a área de \textit{sistemas completamente integraveis}.

\begin{thm}[de Poisson, 1809]\leavevmode
O colchete de Poisson de duas integrais primeiras é uma integral primeira, i.e.
\[\{H,f\}=0,\{H,g\}=0 \implies \left\{ \{f,g\},H \right\} =0\]
\end{thm}

Mas foi o Jacobi que descobreu o meolho daquele teorema:
\begin{thm}[Jacobi, 1842]\leavevmode
\[\{h,\{f,g\}\} +\{g,\{h,f\}\} +\{f,\{g,h\}\} =0\]
\end{thm}
De fato, o teorema de Poisson segue do teorema de Jacobi.

Em 1880 S. Lie trabalha em \textit{álgebras de Lie}. Ele mostra que toda álgebra de Lie vem de um grupo de Lie localmente. Isso levou à \textit{teoria de Lie} onde foram definidas as estruturas de Poisson.

Em 1970, Kirillov, Sourieau e Kostant voltam a trabalhar no colchete de Poisson. Dois artigos que marcaram a era moderna da geometria de Poisson são Lichnerowicz (?) e Weinstein (1983).

\subsection{Motivação para a geometria de Poisson}

\begin{itemize}
\item Mecânica geometrica (plasmas)/Teoria de campos.
\item Sistemas integráveis \(\to\) sistemas bihamiltonianos Poisson-Nijenhuis (algum scenário onde temos duas estruturas de Poisson compatíveis…).
\item Teoria de representações/grupos quânticos (Drinfeld, Fadeev)\(\rightsquigarrow \) grupos de Lie/Poisson.
\item Quantização por deformação. Passagem do formalismo clássico com o colchete de Poisson, para um formalismo com uma álgebra não comutativa \(\mathbb{A},*\).
\end{itemize}

\subsection{Definições}

\begin{defn}\leavevmode
\(M\) variedade diferenciável. Um \textit{\textbf{colchete de Poisson}} em \(M\) é uma operação \(\mathbb{R}\)-bilinear
\begin{align*}
	\{\cdot ,\cdot \}: C^\infty(M)\times C^\infty(M) &\longrightarrow C^\infty(M)
\end{align*}
tal que
\begin{enumerate}
\item (Antisimetria.) \(\{f,g\}=-\{g,f\}\).
\item (Jacobi.) \(\{h,\{f,g\}\} +\{g,\{h,f\}\} +\{f,\{g,h\}\} =0\).
 \item (Leibniz.) \(\{f,gh\}=\{f,g\}h+\{f,h\}g\).
\end{enumerate}
\end{defn}
\begin{remark}\leavevmode
As condicões (1) e (2) dizem que  \((C^\infty(M),\{\})\) é uma álgebra de Lie. Como ainda é uma álgebra comutativa com o produto usual, a condição (3) diz como é que interagem esses dois produtos.
\end{remark}
\begin{remark}\leavevmode
Para qualquer álgebra comutativa \(\mathcal{A}\) podemos introduzir um colchete de Lie e pedir a condição Leibniz, e isso se chama de \textit{\textbf{álgebra de Poisson}}. E se o colchete satisfaz (1) e (3), se chama de \textit{\textbf{estrutura quase Poisson}}, que já tem um significado geométrico.
\end{remark}
\begin{defn}\leavevmode
\textit{\textbf{Aplicação (ou morfismo) de Poisson}} entre \((M_1,\{\cdot ,\cdot \}_1)\) e \((M_2,\{ \cdot ,\cdot \}_2)\) é \(\varphi:M_1 \to M_2\) que preserva colchetes, i.e. o pullback
\begin{align*}
	\varphi^*: C^\infty(M_2) &\longrightarrow C^\infty(M_1) \\
	f &\longmapsto f \circ \varphi
\end{align*}
preserva colchete no sentido de que
\[\{f,g\}_2 \circ \varphi=\{f \circ \varphi,g \circ \varphi\}_1.\]
\end{defn}
Temos um campo que chamamos de \textit{\textbf{hamiltoniano}} que podemos tirar da condição leibniz do colchete. Isso é porque \(\{f,\cdot \}:C^\infty(M) \to C^\infty(M)\) é uma derivação! Definindo esse campo como \(X_f\) obtemos
\[\{f,g\}=\mathcal{L}_{X_f}f=dg(X_f)=-df(X_g)\]

\begin{remark}\leavevmode
Os colchetes  são locais no sentido de que podemos restringir num aberto.
\end{remark}
\begin{remark}\leavevmode
\(\{f,f\}=0\) (\(f\) é preservada por seu campo hamiltoniano).
\end{remark}
\begin{thing6}{Terminologia}\leavevmode
\begin{itemize}
\item \(f,g \in C^\infty(M)\) estão em  \textit{\textbf{involução}} se \(\{f,g\}=0\).
\item Se \(f\) é tal que \(\{f,g\}=0 \forall g \iff X_f=0\), então \(f\) é dita de \textit{\textbf{Casimir}}. No caso simplético isso não faz muito sentido---todas as funções são Casimir porque a forma simplética é não degenerada. Mas no caso geral isso pode mudar. (What?) Aula 2: as Casimir são constantes ao longo da foleação dada por \(\pi ^\sharp\).
\end{itemize}
\end{thing6}

\begin{example}\leavevmode
\begin{itemize}
\item \((M,\{\cdot ,\cdot \}\equiv0)\).

	Outro de jeito de definir funções \(f_1,\ldots,f_k\) com \(\{f_i,f_j\}=0\) é simplesmente dizer \((M,\{\cdot ,\cdot \})\xrightarrow{F}(\mathbb{R}^k,\{\cdot ,\cdot \} \equiv 0)\) é uma aplicação de Poisson.
\item \(\mathbb{R}^{2n}=\{(q,p)\}\), o colchete de Poisson de Poisson é um colchete de Poisson.
\item Mais geralmente, toda vez que tenha uma variedade simmplética \((M,\omega)\), então temos o campo hamiltoniano, e podemos definir como você já sabe \(\{f,g\}:=\omega(X_g,X_f)=dg(X_f)=-df(X_g)\). É isso é um colchete de Poisson, para vê-lo vai notar que Jacobi é equivalente a \(d\omega=0\).

	Qualquer colchete de Poisson que vem de uma estrutura simplética pode ser escrito como o colchete de Poisson de Poisson em coordenadas locais pelo teorema de Darboux.

\item  Considere o produto de \((M_1,\{\cdot ,\cdot \}_1)\) e \((M_2,\{ \cdot ,\cdot \}_2)\) é \(\varphi:M_1 \to M_2\), \(M_1 \times M_2\). Então
	\[\{f,g\}(x_1,x_2)=\{f(\cdot ,x_2),g(\cdot ,x_2)\}_1(x_1)+\{f(x_1,\cdot ),g(x_1,\cdot )\}_2(x_2).\]
	{\color{2}Exercício!} Mostre que esse é um colchete de Poisson. Ainda, que as projeções são mapas de Poisson e os pullbacks de funções em cada \(M_i\) Poisson-comutam no produto, i.e.
	\[\{p_1^*C^\infty(M_1),p_2^*C^\infty(M_2)\}=0\]
	\[\begin{tikzcd}
	&M_1 \times M_2\arrow[dl,"p_1",swap]\arrow[dr,"p_2"]\\
	M_1&&M_2
	\end{tikzcd}\]
\end{itemize}
\begin{proof}[Solution]\leavevmode
A antisimetria é imediata desde que \(\{\cdot ,\cdot \}_1\) e \({\cdot,\cdot}_2\) são antisimétricos. A condição de Jacobi não é imediata para mim:
\begin{align*}
	\{f,\{g,h,\}\}(x_1,x_2)+\{g,\{h,f\}\}(x_1,x_2)+\{h,\{f,g\}\}(x_1,x_2)&=?
\end{align*}
Bom calculemos
\begin{align*}
\{f,\{g,h\}\}(x_1,x_2)&=\{f(\cdot,x_2),\{g,h\}(\cdot,x_2)\}_1(x_1)+\{f(x_1,\cdot),\{g,h\}(x_1,\cdot)\}_2(x_2)\\
\{g,\{h,f\}\}(x_1,x_2)&=\{g(\cdot,x_2),\{h,f\}(\cdot,x_2)\}_1(x_1)+\{g(x_1,\cdot),\{h,f\}(x_1,\cdot)\}_2(x_2)\\
\{h,\{f,g\}\}(x_1,x_2)&=\{h(\cdot,x_2),\{f,g\}(\cdot,x_2)\}_1(x_1)+\{h(x_1,\cdot),\{f,g\}(x_1,\cdot)\}_2(x_2)
\end{align*}
lo que yo digo es que puedo usar Jacobi en \(\{,\}_1\) si veo que \(\{g,h\}(\cdot,x_2)\) es igual a \(\{g(\cdot,x_2),h(\cdot,x_2)\}_1\). Entonces calculo
\begin{align*}
\{g,h\}(\circ,x_2)&\overset{\operatorname{def}}{=}\{g(\cdot,x_2),h(\cdot,x_2)\}_1(\circ)+\{g(\circ,\cdot),h(\circ,\cdot)\}_2(x_2){\color{2}\ldots}
\end{align*}
Por otro lado, creo que Leibniz sí jala:
\begin{align*}
\{f,gh\}(x_1,x_2)&=\{f(\cdot,x_2),gh(\cdot,x_2)\}_1(x_1)+\text{otro lado}
\end{align*}
mientras que
\begin{align*}
\Big(\{f,g\}h\Big)(x_1,x_2)&=\{f(\cdot,x_2),g(\cdot,x_2)\}_1(x_1)h(x_1,x_2)\ldots
\end{align*}
Em fim, ver que as projeções são mapas de Poisson significa que \(p_1^*\{\cdot,\cdot\}\)
\end{proof}
\item \(S\) variedade, considere uma família suave de estruturas simpléticas, i.e. \(\omega_t \in \Omega^2(S)\) simpléticas. Considere as estruturas de Poisson associadas a cada uma delas, i.e. \(\{\cdot ,\cdot \}_t\). Então pode produzir um colchete em \(M:=S \times \mathbb{R}\) dado por
	\[\{f,g\}(x,t):=\{f(\cdot ,t),g(\cdot ,t)\}_t(x).\]
	Creo que es sólo pegar la estructura de Poisson para cada \(t\).
\end{example}

\subsection{Ponto de vista tensorial}

Para isso precisamos de \textit{\textbf{multivetores}}, que é o análogo das formas diferenciais para campos vetoriais, i.e. seções do fibrado \(\Lambda^{\bullet}(TM)\). I.e., \(\mathfrak{X}^\bullet(M)=\Gamma(\Lambda^{\bullet}(TM))\).

Considere uma estrutura \(\{\cdot ,\cdot \}\) quase-Poisson em \(M\). Então existe um único bivetor \(\pi \in \Gamma(\Lambda^{2}(TM))\) tal que
\[\{f,g\}=\pi(df,dg)\]
O lance é que o colchete quase-Poisson não depende das funções, mas sim das diferenciais delas.

De fato,
\[\left\{ \substack{\begin{array}{c}\text{colchetes quase-P}  \\ \{\cdot ,\cdot \}\end{array}} \right\} \leftrightsquigarrow \left\{ \substack{\begin{array}{c}\text{campos de bivetores em \(M\)}  \\ \pi \in \Gamma(\Lambda^{2}(TM))\end{array}}\right\} \]
Dizemos que \(\pi \in \Gamma(\Lambda^{2}(TM))\) é um \textit{\textbf{bivetor de Poisson}} se \(\{f,g\}=\pi(df,dg)\) satisfaz Jacobi. Em coordenadas locais \((x_1,\ldots,x_n)\), um bivetor \(\pi\) sempre pode ser escrito como
\begin{align*}
\pi&=\frac{1}{2}\sum_{i,j}\pi_{i,j}(x) \frac{\partial }{\partial x_i}\wedge \frac{\partial }{\partial x_j}\\
&=\sum_{i<j}\pi_{ij}(x)\frac{\partial }{\partial x_i}\wedge\frac{\partial }{\partial x_j}
\end{align*}
de modo que qualquer colchete quase-Poisson, localmente se escreve desse jeito:
\[\{f,g\}(x)=\sum_{i,j}\pi_{ij}(x)\frac{\partial f}{\partial x_i}\frac{\partial g}{\partial x_j}\]
onde (creo que), em coordenadas locais, \(\pi_{ij}(x)=\{x_i,x_j\}\).
\begin{remark}\leavevmode
	A condição de ser Jacobi nas funções \(\pi_{ij}\) é uma EDP muito difícil, tem outras formas de mostrar que algo é Jacobi além de resolvê-la.
\end{remark}

\section{Aula 2}

\begin{thing6}{Extra}\leavevmode
Estamos seguindo as notas de H. Burstzyn.
\end{thing6}
\[\left\{ \substack{\begin{array}{c}\text{colchetes quase-P}  \\ \{\cdot ,\cdot \}\end{array}} \right\} \leftrightsquigarrow \left\{ \substack{\begin{array}{c}\text{campos de bivetores em \(M\)}  \\ \pi \in \Gamma(\Lambda^{2}(TM))\end{array}}\right\} \leftrightsquigarrow \left\{\substack{\begin{array}{c}\pi^\sharp:T^*M \to TM, \alpha \mapsto  \pi(\alpha,\cdot), \\ \text{t.q.}  (\pi ^\sharp)^*=-\pi ^\sharp \end{array}}\right\}\]
Além disso, os mapas \(\pi ^\sharp\) são equivalentement pensados como mapas que mandam
\[df \mapsto  X_f\]
\[\alpha \mapsto  \pi(\alpha,\cdot)\]
\begin{exercise}\leavevmode
Para \((M_1,\pi_1)\), \((M_2,\pi_2)\) de Poisson e \(\varphi:M_1 \to M_2\), são equivalentes
\begin{itemize}
\item \(\varphi:M_1\to M_2\) é de Poisson.
\item \(X_{\varphi^*f}\overset{\varphi}{\sim}X_f\) (os campos hamiltonianos são \(\varphi\)-relacionados)
\item O seguinte diagrama comuta
	\[\begin{tikzcd}
		T_xM_1 \arrow[r, "d_x\varphi"]&  T_{\varphi(x)}M_2\\
		T_x^*M_1\arrow[u,"\pi_1"]& T_{\varphi(x)}^* M_2\arrow[l,swap,"(d_x\varphi)^*"]\arrow[u,"\pi_2^\sharp"]
	\end{tikzcd}\]
\end{itemize}
\end{exercise}

O ponto de vista tensioral nos da um jeito de definir a estrutura Poisson no produto mais natural
\begin{thing7}{Terminologia}\leavevmode
\textit{\textbf{Integrabilidade}} significa que \(\{\cdot,\cdot\}\) é Jacobi.
\end{thing7}

\subsection{Jacobiador}

Dado um bivetor (sem condição de integrabilidade ainda) \(\pi \in \mathfrak{X}^2(M)\), considere o colchete de Poisson associado,  \(\{f,g\}=\pi(df,dg)\). Definimos

\begin{align*}
	\operatorname{Jac}_\pi: C^\infty(M)\times C^\infty(M)\times C^\infty(M) &\longrightarrow C^\infty(M) \\
	\operatorname{Jac}_\pi(f,g,h) &=\{f,\{g,h\}\}+\{h,\{f,g\}\}+\{g,\{h,f\}\}
\end{align*}

\begin{exercise}\leavevmode
	\[\operatorname{Jac}(f,g,h)=\mathcal{L}_{[X_f,X_g]}h-\mathcal{L}_{X_{\{f,g\}}}h=(\mathcal{L}_{X_f})(dg,dh)\]
\textbf{Hint} Use a regra geral para derivada de Lie de tensores.
\end{exercise}

Vamos ver que consequências temos.

\begin{thing8}{Consequência}\leavevmode
\(\pi\) é de Poisson \(\iff\)
\begin{itemize}
\item 
	\begin{align*}
		C^\infty(M) &\longrightarrow \mathfrak{X}^1(X)\\
		f &\longmapsto X_f
	\end{align*}
	é um morfismo de álgebras de Lie, i.e. \(X_{\{f,g\}}=[X_f,X_g]\)

\item \(L_{X_f}\pi=0\).

\item Temos um trivetor que vai ser zero \(\iff\) o colchete é Poisson:
	\[\exists ! \gamma_\pi \in \mathfrak{X}^3(M)=\Gamma(\Lambda^{3}TM)\]
	\[\operatorname{Jac}(f,g,h)=\gamma_\pi(df,dg,dh)\]	
	(para achar \(\gamma\) usamos que o Jacobiador só depende das diferenciais das funções).	
\end{itemize}
\end{thing8}

\begin{remark}\leavevmode
Em coordenadas locais, \(\operatorname{Jac}=0\) \(\iff\) \(\operatorname{Jac}(x_i,x_j,x_k)=0\). I.e. como \(\operatorname{Jac}\) é na verdade um tensor, pordemos pegar coordenadas locais e comprovar que é zero nas funções coordenadas.
\end{remark}

\begin{example}\leavevmode
\(U \subset \mathbb{R}^n\) aberto, coordenadas \((x_1,\ldots,x_n\), se um bivetor
\[\pi= \sum_{i <  j}\pi_{ij}\frac{\partial }{\partial x_i}\wedge \frac{\partial }{\partial x_j}\]
(lembre que \(\pi_{ij}=\{x_i,x_j\}\)) é tal que \(\pi_{ij}\) são constantes, automáticamente \(\pi\) é de Poisson, porque \(\{x_k,\{x_i,x_j\}\}=0\) já que é uma derivacão.
\end{example}

\begin{example}[O colchete de Poisson é de Poisson]\leavevmode
Porque
\[\pi_{\operatorname{can}}=\sum_{i=1}^n \frac{\partial }{\partial p_i}\wedge \frac{\partial }{\partial q_i}\]
tem \(\pi_{ij}\) constantes.
\end{example}

\begin{example}[Toda superfície é de Poisson]\leavevmode
Porque todo trivetor é zero.
\end{example}

\begin{example}[Colchetes de Poisson em \(\mathbb{R}^2\) são funções]\leavevmode
Vamos falar mais disso depois. O Arnold tava estudando singularidades de funções desse jeito.
\end{example}

\begin{example}[\(\mathbb{R}^3\)]\leavevmode
Pensemos que \(\mathbb{R}^3=\{\xi=(x,y,z)\}\) e definamos
\[\{f,g\}(\xi):= \left<\xi,\nabla f|_{\xi}\times \nabla |_{\xi}\right>\]
{\color{2}Você está convidado a ver que}
\[\pi=z \frac{\partial }{\partial x}\wedge\frac{\partial }{\partial y}+x \frac{\partial }{\partial y}+ y \frac{\partial }{\partial z}\wedge\frac{\partial }{\partial x}\]
\end{example}

\begin{thing7}{Propriedade geral}\leavevmode
O colchete de Poisson sempre desce para o quociente (quando temos uma ação de um grupo de Lie).

Um exemplo disso é o espaço cotangente de um grupo de Lie, \(T^* G\), onde aparece \(\mathfrak{g}^*\). Sim, porque o fibrado (co)tangente de um grupo de Lie é trivial, então \(T^*G \cong G\times \mathfrak{g}^*\).
\end{thing7}

\subsection{Não temos derivada exterior mas…}
Lembre: \(\mathfrak{X}^k(M)\overset{\operatorname{def}}{=}\Gamma(\Lambda^{k}TM)\). Já temos uma coisa graduada:
\[\mathfrak{X}^\bullet(M)\overset{\operatorname{def}}{=}\bigoplus_{k=0}^{\dim M}\mathfrak{X}^k(M)\]
onde definimos \(\mathfrak{X}^0(M):= C^\infty(M)\).

\begin{defn}[Colchete de Shouten-Nijenhuis]\leavevmode
	\[[\cdot,\cdot]:\mathfrak{X}^k(M) \times \mathfrak{X}^\ell(M) \longrightarrow \mathfrak{X}^{k+ \ell-1}(M)\]
satisfazendo
\begin{itemize}
\item \(\mathbb{R}\)-bilinear
\item \([X,Y]=-(-1)^{(k-1)(\ell-1)}[Y,X]\)
\item \([X,Y \wedge Z]=[X,Y]\wedge Z + (-1)^{(k-1)\ell}Y \wedge[X,Z]\)
\item (Nos campos vetoriais é o colchete de Lie.) \(Z \in \mathfrak{X}^1(M)\), \(\implies\) \([Z,X]=\mathcal{L}_ZX\).
\end{itemize}
\end{defn}

\begin{thm}\leavevmode
Existe um único colchete com essas propriedades. Além disso,
\[(-1)^{(k-1)(m-1)}[X,[Y,Z]+(-1)^{(m-1)(\ell-1)}[Z,[X,Y]]+(-1)^{(\ell-1)(k-1)}[Y,[Z,X]]=0\]
onde \(|X|=k\), \(|Y|=\ell\) e \(|Z|=m\).
\end{thm}

Isso significa que \((\mathfrak{X}^\bullet,[\cdot,\cdot])\) é uma \textit{\textbf{álgebra de Poisson graduada}} (com grau \(-1\)). O Gerstenhaber faz uma teoria de deforma ção de lagneras de Lie onde aparem esses objetos.

\begin{thing6}{Lance}[Extra]\leavevmode
	Pode ver \([\pi,\cdot]\) como uma derivação \(\mathfrak{X}^\bullet \to \mathfrak{X}^{\bullet+1}\) quando \(\pi\) é um bivetor.
\end{thing6}

Bom, queremos usar esse colchete para estudar as estruturas de Poisson de um jeito mais intrínseco.

\begin{exercise}\leavevmode
	Pegue um bivetor \(\pi \in \mathfrak{X}^2(M)\).
	\begin{itemize}
		\item \([\pi,f]=-X_f\) \(\forall  f\in C^\infty(M)\).
		\item (Todo mundo teria a capacidade de fazer essa conta.) Vale:
			\begin{align*}
				\frac{1}{2}[\pi,\pi]&=\gamma_\pi \qquad \text{(o Jacobiador)} 
			\end{align*}
			Não é obvio mas a equação \([\pi,\pi]=0\) é extremamante não trivial porque temos uma loucura de sinais. Mas o ponto é que se  \([\pi,\pi]=0\).
	\end{itemize}
\end{exercise}

\begin{remark}\leavevmode
Temos uma analogia para estruturas complexas, usando o colchete de Neihujaus, que é um colchete que mora em \(\Omega^{\bullet}(M,TM)\) (\(TM\)-valued differential forms). Então quando ele se anula a estrutura quase-complexa vira complexa…
\end{remark}

\begin{remark}\leavevmode
\((x_1,\ldots,x_m)\) coordanadas, \(\xi_i:= \frac{\partial }{\partial x_i}\), \(X \in \mathfrak{X}^k(M)\), \(Y \in \mathfrak{X}^\ell(M)\).
\[X= \sum_{i_1<\ldots<i_k}a_{i_1\ldots i_k}(x)\xi_{i_1}\ldots\xi_{i_k}Y=\sum_{i_1<\ldots<i_k}b_{i_1\ldots i_k}(x)\xi_{i_1}\ldots\xi_{i_k}\]
pensando que isso daqui é só notação…
\[[X,Y]=\sum_{i}\frac{\partial X}{\partial \xi_i}\frac{\partial Y}{\partial x_i}-(-1)^{(k-1)(\ell-1)}\frac{\partial X}{\partial x_i}\frac{\partial Y}{\partial \xi_i}\]
que te lembra do colchete de Poisson original.

\end{remark}



\end{document}
